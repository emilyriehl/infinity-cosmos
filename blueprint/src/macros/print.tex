% In this file you should put macros to be used only by
% the printed version. Of course they should have a corresponding
% version in macros/web.tex.
% Typically the printed version could have more fancy decorations.
% This should be a very short file.
%
% This file starts with dummy macros that ensure the pdf
% compiler will ignore macros provided by plasTeX that make
% sense only for the web version, such as dependency graph
% macros.


% Dummy macros that make sense only for web version.
\newcommand{\lean}[1]{}
\newcommand{\discussion}[1]{}
\newcommand{\leanok}{}
\newcommand{\mathlibok}{}
\newcommand{\notready}{}
% Make sure that arguments of \uses and \proves are real labels, by using invisible refs:
% latex prints a warning if the label is not defined, but nothing is shown in the pdf file.
% It uses LaTeX3 programming, this is why we use the expl3 package.
\ExplSyntaxOn
\NewDocumentCommand{\uses}{m}
 {\clist_map_inline:nn{#1}{\vphantom{\ref{##1}}}%
  \ignorespaces}
\NewDocumentCommand{\proves}{m}
 {\clist_map_inline:nn{#1}{\vphantom{\ref{##1}}}%
  \ignorespaces}
\ExplSyntaxOff

%
\newcommand{\Del}{\mbfDelta}

%arrows; cheap version of \we and \trvfib
\makeatletter
\def\makeslashed#1#2#3#4#5{#1{\mathpalette{\sla@{#2}{#3}{#4}}{#5}}}

\def\@mathlower#1#2#3{\setbox0=\hbox{$\m@th#2#3$}\lower#1\ht0\box0}
\def\mathlower#1#2{\mathpalette{\@mathlower{#1}}{#2}}
\makeatother

\newcommand\dhxrightarrow[2][]{%
  \mathrel{\ooalign{$\xrightarrow[#1\mkern4mu]{#2\mkern4mu}$\cr%
  \hidewidth$\rightarrow\mkern4mu$}}
}


% New style, simpler, inline arrows.
\let\xto=\xrightarrow
\newcommand{\fib}{\twoheadrightarrow}
\newcommand{\we}{\xrightarrow{{\smash{\mathlower{0.8}{\sim}}}}}
\newcommand{\trvfib}{\dhxrightarrow{{\smash{\mathlower{0.8}{\sim}}}}}
\newcommand{\To}{\Rightarrow}%2-cells
\newcommand{\isoto}{\xrightarrow{{\smash{\mathlower{0.8}{\cong}}}}}
\newcommand{\inc}{\hookrightarrow}

\newcommand{\catfour}{{\Bbbfour}}%texdoc unimath-symbols.pdf
\newcommand{\catthree}{{\Bbbthree}}%texdoc unimath-symbols.pdf
\newcommand{\cattwo}{{\Bbbtwo}}
\newcommand{\catone}{{\Bbbone}}
\newcommand{\catn}{{\mathbb{n}}}
\newcommand{\catnone}{{\catn\!+\!\catone}}
\newcommand{\catntwo}{{\catn\!+\!\cattwo}}
\newcommand{\iso}{{\BbbI}}
